\begin{frame}
\frametitle{Продвинутые алгоритмы аллокации памяти}
\framesubtitle{Аллокация в несколько этапов}

Современные аллокаторы памяти выделяют две стадии:

\begin{itemize}
  \item<2-> аллокация больших блоков (Buddy Allocator и Ко.):
    \begin{itemize}
      \item аллокации просиходят нечасто, большие объекты живут долго
      \item чем больше блок тем меньше накладные расходы на служебные структуры алокатора - можем хранить больше информации
    \end{itemize}
  \item<3-> аллокация маленьких блоков фиксированного размера (SLAB и Ко.):
    \begin{itemize}
      \item блоки фиксированного размера проще аллоцировать
      \item блоки фиксированного размера требуют меньше служебной информации
      \item блоки имеют одинаковый размер не случайно - часто это объекты одного типа и это можно использовать
    \end{itemize}
\end{itemize}

\end{frame}

\begin{frame}
\frametitle{Buddy Allocator}
\framesubtitle{Вводные положения}

\begin{itemize}
  \item вся аллоцируемая память разбита на большие блоки фиксированного размера (будем называть их PAGE), все PAGE-ы пронумерованы
  \item память аллоцируется и освобождается блоками по $2^i\times PAGE$
\end{itemize}
\end{frame}
