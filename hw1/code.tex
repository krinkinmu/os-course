\section{Предоставляемый код}

Вместе с заданием мы предоставляем вам начальный код и билд скрипт. Вы можете
вносить в него любые измения, пытать его как вам угодно или вообще выкинуть и
не использовать. Назначение этого кода упростить вам жизнь, чтобы вам не
пришлось сразу и неожиданно писать целую кучу\footnote{в этом задании вам,
скорее всего придется написать какое-то количество ассемблерного кода, но это
будет уже не целая куча.} кода на языке ассемблера, с которым вы не знакомы, но
это не значит, что вы можете вообще не разбираться в этом коде, потому что я не
всегда буду таким добрым и когда-нибудь вам придется столкнуться с суровой
реальностью.

Пройдемся по списку:
\begin{itemize}
  \item Makefile - тут трудно что-то добавить, он довольно примитивный,
        добавлять в него свои файлы не проблема;
  \item videomem.S - код для работы с видеопамятью, он нужен только для
        bootstrap.S пока вы не настроили последовательный порт, после этого
        работа с видеопамятью нам не понадобится;
  \item bootstrap.S - в этом файле находится точка входа в программу, он
        отвечает за настройку начальной GDT, инициализацию paging-а (о которой
        вам не нужно волноваться на данном этапе) и перевод процессора в так
        называемый long mode (64-битный режим работы), после чего он просто
        вызывает функцию main (с которой вы и начнете);
  \item kernel.ld - линкер скрипт, из этого файла вы можете увидеть, что ядро
        загружается начиная с 1M физической памяти, но расчитано на работу в
        "верхах" виртуального адресного пространсва (а именно c адреса
        0xffffffff80000000), таким образом все нижнее виртуальное адресное
        простраснвто будет свободно для userspace приложений;
  \item memory.h - файл, который содержит информацию о конфигурации памяти,
        в частности он содержит используемые начальной GDT селекторы кода и
        селекторы данных (KERNEL\_CODE и KERNEL\_DATA, соответсвенно);
  \item ioport.h - функции работы с портами ввода/вывода, которые наверняка вам
        пригодятся в домашнем задании;
  \item interrupt.h - содержит описание структуры-указателя на IDT и функцию
        для установки значения IDTR, они тоже могут оказаться полезными;
  \item kernel\_config.h - предполагается, что вы сюда будете добавлять
        различные опции конфигурации, на данный момент там есть только одна
        опция, о которой будет рассказано в разделе про использование QEMU;
  \item main.c - содержит функцию main, которая не делает ничего полезного.
\end{itemize}
