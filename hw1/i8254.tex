\section{Программируемый интервальный таймер (PIT)}

Как и для контроллера прерываний для интервального таймера есть спецификация,
которая обладает тем же недостатком для нас, что и спецификация контроллера
прерываний~\cite{INTEL:8254DS}. Поэтому в данном разделе я расскажу про PIT
подробнее.

PIT внутри содержит генератор тактов который генерирует сигналы с частотой
1193180 Гц. Выбрав нужный режим работы и коэффициент деления вы можете с его
помощью генерировать прерывания через равные небольшие интервалы времени.

Подключен PIT обычно к IRQ0 (т. е. нулевой ноге Master контроллера), а значит,
если вы отобразили аппаратные прерывания на вектора начиная с 0x20, то номер
вектора прерывания PIT будет 0x20.

К счастью PIT не требует какой-то сложной настройки. Для настройки PIT так же
как и для Legacy PIC используются порты ввода/вывода. В частности Control Port
имеет номер 0x43 и отвечает за команду и выбор "канала" и Data Port, который для
каждого "канала" свой, в частности для нулевого (который вам и предлагается
использовать) это порт 0x40.

Упомянутые выше "каналы" вы модете рассматривать как различные таймеры внутри
PIT. Всего каналов 3, но два из них условно заняты, а для свободного
использования остается только нулевой канал, с которым мы и будем работать.

Вы уже должны примерно представлять как работают различные таймеры в общем, PIT
мало чем от них отличается, так что большая часть из того что будет расписано
дальше может выглядеть подозрительно знакомой.

Начнем с режимов работы, их 6, как всегда нам нужен только один - Rate Generator
\footnote{Вообще говоря нам подойдут два режима из 6 - можно также использовать
Square Wave Mode.}. Как не трудно догадаться (по названию и из задания, которое
вам нужно сделать) в этом режиме PIT генерирует прерывания через равные
интервалы времени.

Коэффициент деления это 16-битное число. И это все что можно сказать про
коэффициент деления. Чтобы установить режим работы и коэффициент деления нужно
записать в Control Port команду в специальном формате:

\begin{itemize}
  \item bit 0 - отвечает за выбор формата числа, в котором мы передаем делитель
        частоты, 0 значит, что используется обычное двоичное кодирование, а
        1 значит, что используется BCD\footnote{Не уверен, что нужно говорить,
        но вы наверняка не хотите использовать BCD};
  \item bits [3:1] - эти 3 бита содержат номер режима работы, для Rate Generator
        используется значение 2;
  \item bits [5:4] - определяют какую часть коэффициента деления мы хотим
        записать, 1 значит, что мы хотим записать только наимнее значимый байт
        из двух, 2 значит, что мы хотим записать наиболее значимый из двух, ну
        и наконец 3 значит, что первая операция записи запишет наименее
        значимый байт, а вторая операция записи запишет наиболее значимый байт
        \footnote{Так как нам надо инициализировать PIT только один раз, вы
        скорее всего должны использовать значение 3.};
  \item bits[7:6] - выбирают какой из каналов вы настраиваете, так как мы
        работаем только с нулевым, то значение этих бит, очевидно, 0;
\end{itemize}

Далее необходимо записать в Data Port значение коэффициента деления. В
зависимости от значения bits [5:4] вам потребуется сделать либо одну запись
либо две записи подряд.
