\begin{frame}
\frametitle{Tagged Pointer}

Не все биты указателя реально используются:
\begin{itemize}
  \item например, данные в динамической памяти зачастую аллоцируются выпровнеными на границу $2^i$, где $i$ обычно небольшое число (4-5);
  \item в x86-64, под адресацию используется только 48 бит, оставшиеся 16 либо 0 либо 1 (канонический адрес);
  \item мы можем создавать свои собственные "указатели" разного размера (увидим дальше в примере);
  \item<2-> мы можем использовать эти неиспользуемые биты!
\end{itemize}
\end{frame}

\begin{frame}[fragile]
\frametitle{Tagged Pointer}

\begin{lstlisting}
typedef std::uint16_t   StackNodePtr;
typedef std::uint16_t   StackNodeTag;
typedef std::uint32_t   StackHeadPtr;
\end{lstlisting}

Заведем свой "внутренний" указатель:
\begin{itemize}
  \item StackNodePtr - 16-битное число, которое будет выступать в качестве указателя элемента стека;
    \begin{itemize}
      \item что вообще можно адресовать 16-битным числом?
      \item<2-> все что угодно, если этих объектов не больше чем $2^{16}$
    \end{itemize}
  \item<3-> StackNodeTag - 16-битное число, служебная информация, которую мы храним рядом с указателем элемента;
  \item<4-> StackHeadPtr - объединение StackNodePtr и StackNodeTag в одно 32-битное число.
\end{itemize}
\end{frame}

\begin{frame}[fragile]
\frametitle{Tagged Pointer}

Функции для работы с StackNodePtr, StackNodeTag и StackHeadPtr:
\begin{lstlisting}
static StackHeadPtr createPtr(StackNodePtr ptr, StackNodeTag tag)
{ return ptr | ((StackHeadPtr)tag << 16); }

static StackNodePtr getPtr(StackHeadPtr ptr)
{ return ptr & 0xffff; }

static StackNodeTag getTag(StackHeadPtr ptr)
{ return ptr >> 16; }
\end{lstlisting}
\end{frame}

\begin{frame}[fragile]
\frametitle{Tagged Pointer}

\begin{lstlisting}
struct StackNode {
    StackNodePtr next;
    T *data;

    StackNode(T *data = nullptr) : next(0), data(data) { }
};

StackNode *                     pool;
std::atomic<StackHeadPtr>       head;
std::atomic<StackHeadPtr>       free;
\end{lstlisting}

\begin{itemize}
  \item Изменим определение полей нашего стека, чтобы они использовали StackHeadPtr и StackNodePtr;
  \item Чтобы StackNodePtr "указывал" на реальный объект в памяти заведем массив pool
    \begin{itemize}
      \item StackNodePtr будет индексом в этом массиве;
    \end{itemize}
  \item Для аллоции и освобождения элементов стека динамически используем список free;
\end{itemize}
\end{frame}

\begin{frame}[fragile]
\frametitle{Аллокация узлов стека}
\begin{lstlisting}
StackNodePtr allocatePtr()
{
    StackHeadPtr head, next;
    StackNodePtr ptr;

    do {
        head = free.load(std::memory_order_relaxed);
        ptr = getPtr(head);

        if (!ptr)
            continue;

        StackNode *node = getNode(ptr); // &pool[ptr]
        next = createPtr(node->next, getTag(head) + 1);
    } while (!free.compare_exchange_strong(head, next));

    return ptr;
}
\end{lstlisting}
\end{frame}

\begin{frame}[fragile]
\frametitle{Освобождение узлов стека}

\begin{lstlisting}
void freePtr(StackNodePtr ptr)
{
    StackHeadPtr head, new_head;
    StackNode *node = getNode(ptr);

    do {
        head = free.load(std::memory_order_relaxed);
        node->next = getPtr(head);
        new_head = createPtr(next, getTag(head) + 1);
    } while (!free.compare_exchange_strong(head, new_head));
}
\end{lstlisting}
\end{frame}

\begin{frame}[fragile]
\frametitle{Реализация push}

\begin{lstlisting}
template <typename T>
void Stack<T>::push(T *value)
{
        StackHeadPtr old_head, new_head;
        StackNodePtr ptr = allocatePtr();
        StackNode *node = getNode(ptr);
        node->data = value;
        do {
                old_head = head.load(std::memory_order_relaxed);
                new_head = createPtr(ptr, getTag(old_head) + 1);
                node->next = getPtr(old_head);
        } while (!head.compare_exchange_strong(old_head, new_head));
}
\end{lstlisting}
\end{frame}

\begin{frame}[fragile]
\frametitle{Реализация pop}

\begin{lstlisting}
template <typename T>
T *Stack<T>::pop()
{
        StackHeadPtr old_head, new_head;
        StackNodePtr ptr;
        StackNode *node;

        do {
                old_head = head.load(std::memory_order_relaxed);
                ptr = getPtr(old_head);
                if (!ptr)
                        return nullptr;
                node = getNode(ptr);
                new_head = createPtr(node->next, getTag(old_head) + 1);
        } while (!head.compare_exchange_strong(old_head, new_head));

        T *data = node->data;
        freePtr(ptr);
        return data;
}
\end{lstlisting}
\end{frame}
