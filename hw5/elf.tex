\section{Загрузка и запуск ELF файла}

В задании вам не нужно уметь загружать и запускать любые файлы формата ELF,
достаточно, если вы сможете предоставить исходники и скрипт для сборки ELF
файла, который вы можете запустить.

Сама по себе загрузка ELF не сложный процесс, если у вас уже есть файловый
интерфейс, функции алокации страниц и функции для работы с таблицей
страниц;

Все что вам нужно, это прочитать таблицу программных заголовков, найти те
из них, тип которых \emph{PT\_LOAD}, настроить в таблице страниц отображение,
которое покроет диапазон от \emph{p\_vaddr} до \emph{p\_vaddr + p\_memsz} и
скопировать в этот диапазон память участок файла начиная со смещения
\emph{p\_offset} и до \emph{p\_offset + p\_filesz}. На этом загрузка файла
заканчивается. Определения нужных структур и констант вы можете найти в файле
\emph{elf.h} в папке \emph{src}.

Запуск этого файла в пространстве пользователя требует меньше кода, но
может быть концептуально более сложной задачей. Для этого вам необходимо
создать на стеке структуру, которая будет прочитана из памяти инструкцией
\emph{iret}, которую мы будем использовать для передачи управления в
пространство пользователя.

Эта структура должна включать в себя селекторы стека и селекторы кода
пользовательского пространства, указатель стека и указатель команд, а также
регистр флагов. Все они будут записаны инструкцией \emph{iret} в
соответствующие регистры процессора и таким образом управление будет
преданно в пространство пользователя.

В качестве селекторов стека и кода используйте \emph{USER\_DATA} и
\emph{USER\_CODE} из файла \emph{memory.h}. А в качестве указателя команд
используйте точку входа указанную в заголовке ELF файла. Указатель стека
остается на ваше усмотрение, например, вы можете возложить настройку стека
на пользовательское приложение, или в ядре выделить память под стек где-нибудь
вверху пользовательского адресного пространства (ниже дыры в каноническом
адресном пространстве).
