\begin{frame}
\frametitle{О чем курс?}

\begin{itemize}
  \item<1-> Memory Management:

    \begin{itemize}
      \item различные алгоритмы аллокации (физическая память, виртуальная
            память, SLOB/SLAB/SLUB?)
      \item paging (магия page fault, вроде бы fault, а вроде и нет)
      \item интерфейс ОС для userspace процессов
    \end{itemize}

  \item<2> Multithreading and Scheduling:
    \begin{itemize}
      \item критерии и алгоритмы планирования, переключение тредов
      \item особенности многоядерных архитектур, когерентность кешей и барьеры
            памяти
      \item синхронизация потоков исполнения
    \end{itemize}
\end{itemize}

\end{frame}

\begin{frame}
\frametitle{О чем курс?}

\begin{itemize}
  \item<1-> Inter Process Communication

  \item<2-> Persistent Storage:
    \begin{itemize}
      \item локальные файловые системы (пожалуй самая интересная тема курса, тут
            мы даже посмотрим на пару нормальных алгоритмов)
      \item распределенные файловые системы и связанные темы (консенсус, CAP,
            неразрешимые проблемы и пр.)
    \end{itemize}

  \item<3> Virtualization:
    \begin{itemize}
      \item кому и зачем это нужно
      \item немного о внутреннем устройстве (trap and emulate)
    \end{itemize}

\end{itemize}

\end{frame}
