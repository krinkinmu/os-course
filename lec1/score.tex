\begin{frame}
\frametitle{Как вас будут оценивать?}

\begin{itemize}
  \item<1-> Лекции? \onslide<2->{Я не отмечаю посещаемость и не устраиваю
            контрольных.} \onslide<3->{Пока мне не дадут по голове.}
  \item<4-> Домашние задания:
    \begin{enumerate}
      \item<5-> прерывания
      \item<6-> аллокация памяти
      \item<7-> файловые системы
      \item<8-> планирование потоков и процессы
      \item<9-> userspace и системные вызовы
    \end{enumerate}
\end{itemize}
\end{frame}

\begin{frame}
\frametitle{Как вас будут оценивать?}
\framesubtitle{Оценка домашних заданий.}

\begin{itemize}
  \item<1-> Каждое задание состоит из основных подзадач и дополнительных
            подзадач. Каждая подзадача +1 к количеству баллов за задание.
  \item<2-> Для получения отличной оценки достаточно выполнять только
            обязательные подзадачи.
  \item<3-> Ни одно задание само по себе не влияет на итоговую оценку:
    \begin{itemize}
      \item<4-> полностью завалили одно любое задание - все еще можно получить
                "отлично"
      \item<5-> завалили два - все еще можете рассчитывать на "хорошо"
      \item<6-> завалили три - все еще можете ждать "удовлетворительно"
      \item<7-> ну вы поняли...
    \end{itemize}
\end{itemize}
\end{frame}

\begin{frame}
\frametitle{Как вас будут оценивать?}
\framesubtitle{Альтернативы.}

\onslide<1->{Если у вас есть идеи того чего бы вы хотели сделать относящегося к
ОС, то вы можете делать этот проект вместо домашних заданий. Но...}
\onslide<2->{Не любой проект подойдет - нужно согласовать!}
\end{frame}
