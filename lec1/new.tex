\begin{frame}
\frametitle{70-ые}

Кроме UNIX в 70-ых годах было еще несколько довольно интересных моментов:

\begin{itemize}
  \item 1971 - Intel и ее микропроцессоры
  \item персональные компьютеры (Altair и потом Apple)
  \item IBM VM - компания IBM разработала модификацию IBM System/360 с
        поддержкой трансляции памяти и ОС систему для нее (the Virtual Machine
        Operating System), так что каждый процесс получал свое независимое
        адресное пространство; главная цель была в возможности запускать
        несколько ОС на одной машине одновременно
\end{itemize}
\end{frame}

\begin{frame}
\frametitle{80-ые}

\begin{itemize}
  \item IBM PC 1981 (то что мы сейчас называем персональными компьютерами) и
        PC-DOS от Microsoft
  \item сетевая файловая система NFS появляется в 1985 году (тогда компания Sun
        была в рассвете)
  \item Mach, 1986 - микроядерный подход к разработке ОС (так и не добилась
        успеха, и так ей и надо)
  \item Plan 9 - UNIX сначала, теперь распределенная версия (так и не добилась
        успеха, а жаль)
\end{itemize}
\end{frame}

\begin{frame}
\frametitle{90-ые}

\begin{itemize}
  \item первая успешная версия Windows 3.0 (как вы догадались две другие
        не могли похвастаться большим успехом)
  \item первая успешная Windows - первые вирусы для Windows (так и липнет к ней
        всякая зараза)
  \item Linux, 1991 - кто бы знал тогда, что она так выстрелит
        (\href{https://en.wikipedia.org/wiki/Tanenbaum\%E2\%80\%93Torvalds_debate}{Таненбаум точно не знал})
        (\href{https://www.kernel.org/pub/linux/kernel/Historic/v0.99/}{сравнительно ранние исходники Linux - никакой магии})
  \item Windows NT, 1993 - \href{https://en.wikipedia.org/wiki/Dave_Cutler\#Attitude_towards_Unix}{Дэйв Катлер против UNIX}
\end{itemize}
\end{frame}

\begin{frame}
\frametitle{Что дальше?}
А потом появилась она - <вставьте название, которое вы дадите своей ОС> и
затмила всех.
\end{frame}
