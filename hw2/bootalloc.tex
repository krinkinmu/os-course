\section{Boot Allocator}

Если вы решите использовать одну из модификация Buddy Allocator-а, то вам,
вероятно, понадобится еще один аллокатор для аллокации служебной информации,
чтобы инициализировать Buddy Allocator. Ситуация усугубляется тем, что вы к
этому моменту еще не успеете загрузить свою полноценную таблицу страниц
\footnote{Для того, чтобы ее создать, скорее всего, нужен аллокатор страниц,
которого еще нет.}.

Проблема однако решается довольно просто - вам нужен еще один, очень простой
аллокатор, который будет аллоцировать участки физической памяти из тех областей,
для которых уже есть отображение в начальной таблице страниц (на первые 4GB
физической памяти у вас отображены два участка виртуальной - вам предлагается
использовать участок сразу за "канонической дырой") используя информацию из
карты памяти.

На этот аллокатор не налагается серьезных ограничений, т. е.:

\begin{itemize}
  \item вы можете ограничить количество объектов, которые этот аллокатор может
        аллоцировать в принципе (хранить информацию о них в статическом массиве);
  \item вам не обязательно поддерживать освобождение (память нужная Buddy
        Allocator-у никогда не будет освобождена);
\end{itemize}

Однако вам необходимо проследить, чтобы аллоцированные участки памяти были
зарезервированы (так же как это должно быть сделано для памяти ядра), чтобы
никто больше не мог их использовать.
