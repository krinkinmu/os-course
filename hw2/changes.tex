\section{Изменения в предоставляемом коде}

В предоставленный для первого задания код были внесенные кое-какие изменения:

\begin{itemize}
  \item в kernel.ld было добавлена память под начальную таблицу страниц - если
        вы не вносили изменений kernel.ld, то конфликтов быть не должно;
  \item bootstrap.S - identity mapping и виртуальная память после "канонической
        дыры" теперь отображены на первые 4GB физической памяти;
  \item в файл memory.h были добавлены функции pa и va, которые позволяют
        получить по виртуальному адресу из участка сразу после "канонической
        дыры" физический адрес и наоборот.
\end{itemize}

Кроме того к коду был добавлен еще один заголовочный файл - paging.h. Этот файл
содержит следующие функции:

\begin{itemize}
  \item pte\_preset, pte\_user, pte\_write и pte\_large - проверяют, установлен
        ли соответствующий бит в записи таблицы страниц (бит описываются
        соответствующими define-ами);
  \item pte\_phys - достает из записи таблицы страниц физический адрес таблицы
        следующего уровня или страницы памяти, на которую отображен
        соответствующий виртуальный адрес;
  \item pml\*\_i - функции позволяющие получить индекс в таблице
        соответсвующего уровня (короче говоря, они парсят виртуальный адрес);
  \item page\_off - возвращает по виртуальному адресу смещение внутри 4KB
        страницы (оставшаяся часть парсинга виртуального адреса);
  \item canonical и linear - преобразовывают виртуальный адрес в не каноническом
        формате в канонический, и наоборот;
  \item load\_pml4 - загружается в cr3 переданный физический адрес;
  \item store\_pml4 - возвращает значение записанное в cr3;
  \item flush\_tlb\_addr - сбрасывает запись TLB соответствующую переданному
        виртуальному адресу;
  \item flush\_tlb - сбрасывает все записи в TLB (просто перезаписывает cr3
        старым значением).
\end{itemize}
