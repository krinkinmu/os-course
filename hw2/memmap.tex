\section{Карта памяти}

Получение карты памяти - это чисто формальная часть задания. На самом деле
карта памяти предоставляется multiboot загрузчиком. Те кто не использует
предоставленный bootstrap.S должны будут разобраться с получением карты памяти
самостоятельно. Для всех остальных эта инструкция.

\subsection{Получение карты памяти}

Собственно, bootstrap.S уже получает \emph{физический} адрес multiboot
information structure, формат которого приведен в~\cite{multiboot} (раздел 3.3
Boot information format), вам остается только проверить, что поля mmap\_length и
mmap\_addr валидны\footnote{Для этого разберитесь с описанием поля flags} и
распарсить и использовать эту информацию.

Чтобы получить адрес multiboot information structure добавьте в свой код
следующие строки:

\begin{lstlisting}
extern const uint32_t mboot_info;
\end{lstlisting}

После этого вы можете использовать mboot\_info - в этой переменной и содержится
адрес multiboot information structure. Обратите внимание, на тип - uint32\_t:
\begin{itemize}
  \item вам нужно подключить заголовок stdint.h, чтобы использовать этот тип;
  \item адерс 32-битный, т. е. находится в первых 4GB физической памяти.
\end{itemize}

Несмотря на то, что адрес multiboot information structure физический мы можем
преобразовать его к указателю, так как у нас есть настроенный identity mapping
в начальной таблице страниц, который покрывает первые 4GB памяти.

\subsection{Резервирование памяти ядра}

Карта памяти получается (условно) от аппаратуры компьютера, которая ничего не
знает о ядре ОС. Соответственно, в этой карте памяти не отображено, что часть
памяти занята ядром ОС - вам необходимо зарезервировать эту память
самостоятельно. В противном случае ваши аллокаторы могут ее аллоцировать и вы
перезапишите уже занятую память.

Для того чтобы этого не случилось вам нужно получить границы памяти занятой
ядром ОС, для этого добавьте в код следующие строки:

\begin{lstlisting}
extern char text_phys_begin[];
extern char bss_phys_end[];
\end{lstlisting}

После чего вы можете преобразовать text\_phys\_begin в 64 битное число\footnote{
На самом деле старшие 32 бита должны быть нулевыми.} которое содержит физический
адрес начала вашего ядра, а если вы преобразуете bss\_phys\_end в число, то
получите адрес конца ядра.
