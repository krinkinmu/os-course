\section{Основное задание}

В этом домашнем задании вам необходимо реализовать аллокаторы памяти, которые
вы будете использовать в дальнейших домашних заданиях - очень не желательно
пропускать это задание, а кроме того учите, что вам придется пользоваться тем
интерфейсом, который вы для себя создадите.

Для выполнения этого задания вам, скорее всего, придется настроить Paging под
свои нужды, рекомендации по тому как это сделать даны после задания. Кроме того
вам \emph{настоятельно} рекомендуется прочитать внимательно раздел об изменениях
внесенных в предоставляемые вам файлы, чтобы вы могли разрешить конфиликты
слияния без проблем.

\begin{enumerate}
  \item Получить и вывести карту физической памяти. Добавьте в карту памяти
        участок физической памяти занятый вашим ядром.
  \item Реализовать аллокатор страниц физической памяти. Для аллокации вы можете
        воспользоваться одним из известных алгоритмов (Buddy Allocator или его
        модификации) или придумать свой алгоритм. Какой бы вариант вы не
        выбрали он должен удовлетворять следующим требованиям:
        \begin{itemize}
          \item вы должны уметь аллоцировать память как минимум постранично;
          \item страница должна быть выровнена по размеру страницы, т. е.
                недостаточно просто вернуть любой блок памяти размером в 4KB
                (2MB);
          \item страница должна возвращается пользователю целиком, т. е. нельзя
                хранить служебную информацию аллокатора в аллоцированной
                странице;
          \item страницы нужно уметь освобождать (очевидно);
        \end{itemize}
  \item Реализовать аллокатор блоков фиксированного размера (что-то похожее на
        SLAB~\cite{Bonwick:SLAB}). Опять же вы можете выбирать любой вариант
        реализации, какой захотите, пока он удовлетворяет следующим требованиям:
        \begin{itemize}
          \item вы должны уметь создавать аллокатор блоков фиксированного
                размера для любого размера в пределах от 1B до 4KB
                (не включительно);
          \item аллокатор должен возвращать указатель готовый к использованию;
          \item должна быть предусмотрена возможность выравнивания аллоцируемых
                объектов по указанной при создании аллокатора границе;
          \item указатель должен быть в верхней части адресного пространства,
                т. е. после "канонической дыры";
          \item память нужно уметь освобождать (очевидно);
        \end{itemize}
\end{enumerate}

\section{Дополнительные задания}

\begin{itemize}
  \item Реализовать аллокатор памяти общего назначения, т. е. с интерфейсом на
        подобии malloc/free (названия функций можете выбирать на свое
        усмотрение);
  \item Реализовать функции отображения и удаления отображения физических страниц
        в последовательные участки виртуального адресного пространства, т. е.
        отображение непоследовательных физических страниц в последовательную
        виртуальную память; При этом должны быть выполнены следующие требования:
        \begin{itemize}
          \item диапазон виртуальных адресов должен выбираться автоматически, а не
                передаваться как параметр;
          \item нельзя использовать виртуальные адреса меньше "канонической дыры";
        \end{itemize}
\end{itemize}
