\begin{frame}
\frametitle{Эффективная виртуализация}
\begin{itemize}
  \item<1-> Итак нам нужна виртуализация, как сделать ее эффективной?
    \begin{itemize}
      \item Как вообще понимать эффективность виртуализации?
      \item Когда возможна эффективная виртуализация?
      \item Интуитивно, при эффективной виртуализации как можно больше кода
            должны исполняться нативно (без интерпретации);
    \end{itemize}
  \item<2-> В 74 году два товарища - Попек и Голдберг, потрудились
        сформулировать формальный критерий:
    \begin{itemize}
      \item оригинальная статья "Formal Requirements for Virtualizable Third
            Generation Architectures";
      \item как и для любого формализма нам потребуется несколько вводных
            определений;
    \end{itemize}
\end{itemize}
\end{frame}

\begin{frame}
\frametitle{Модель системы}
\begin{itemize}
  \item<1-> В системе существуют два режима работы - привилегированный и
        непривилегированный
    \begin{itemize}
      \item в привилегированном режиме можно исполнять любые инструкции;
      \item в непривилегированном некоторые инструкции приводят к генерации
            исключения (a. k. a. \emph{trap});
      \item таким образом инструкции делятся на привилегированные и
            непривилегированные;
    \end{itemize}
  \item<2-> Среди инструкций также выделяют чувствительные инструкции:
    \begin{itemize}
      \item управляющие инструкции изменяют конфигурацию ресурсов системы
            (например, изменение таблицы страниц);
      \item инструкции чувствительные к конфигурации инструкции;
    \end{itemize}
\end{itemize}
\end{frame}

\begin{frame}
\frametitle{Критерий виртуализуемости}
\begin{itemize}
  \item<1-> Эффективный \emph{VMM} может быть построен для заданной архитектуры
        при условии, что все чувствительные инструкции являются
        привилегированными
    \begin{itemize}
      \item все инструкции, которые мы хотели бы обрабатывать особым образом
            генерируют \emph{trap};
    \end{itemize}
  \item<2-> \emph{VMM} организован следующим образом:
    \begin{itemize}
      \item \emph{VM} работает в непривилегированном режиме как обычный процесс;
      \item при попытке исполнить "опасную" инструкцию генерируется \emph{trap};
      \item \emph{VMM} перехватывает управление и "эмулирует" инструкцию;
    \end{itemize}
\end{itemize}
\end{frame}
