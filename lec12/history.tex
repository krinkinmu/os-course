\begin{frame}
\frametitle{История вопроса}
\begin{itemize}
  \item<1-> Представим, что вам нужно разработать новую ОС
    \begin{itemize}
      \item вам потребуется отлаживать ОС, что не всегда возможно с
            реальным железом (его может не быть, оно может быть дорогим);
      \item вам поможет симулятор - эмулирующий реальное оборудование;
      \item симуляция очень сильно замедляет исполнение;
    \end{itemize}
  \item<2-> Что если эмулируемая платформа (\emph{guest}) и платформа, на
        которой выполняется эмуляция (\emph{host}) совпадают?
    \begin{itemize}
      \item Можно ли ускорить симуляцию в этом случае?
      \item На сколько эффективной может быть такая эмуляция?
    \end{itemize}
\end{itemize}
\end{frame}

\begin{frame}
\frametitle{История вопроса}
\begin{itemize}
  \item<1-> А зачем вообще эмуляция, если у нас уже есть нужное железо?
    \begin{itemize}
      \item пользовательское программное обеспечение полагается на ОС;
      \item хочется запускать ПО для разных ОС на одной машине;
      \item т. е. хотим запускать несколько ОС на одной машине.
    \end{itemize}
  \item<2-> Виртуальные машины изначально были разработаны для решения описанной
        проблемы
    \begin{itemize}
      \item гиппервизор (Virtual Machine Monitor) - создает иллюзию нескольких
            экземпляров одного оборудования;
      \item виртуальная машина (Virtual Machine) - изолированное окружение для
            запуска экземпляра ОС;
    \end{itemize}
\end{itemize}
\end{frame}
